\documentclass[]{article}
\usepackage{fullpage}
\usepackage[authoryear]{natbib}
\usepackage{setspace}
    \doublespacing
\usepackage{hyperref}
\hypersetup{
    colorlinks,
    citecolor=black,
    filecolor=black,
    linkcolor=cyan,
    urlcolor=cyan
}
\usepackage{amssymb,amsmath}
\usepackage{bm}
\usepackage{dcolumn}
\usepackage{booktabs}
\usepackage{url}
\usepackage{tikz}
\usepackage{todonotes}
\usepackage[utf8]{inputenc}
\usepackage{graphicx}
\usepackage{longtable}
\usepackage{todonotes}
\usepackage{lscape}
\usepackage{float}


\title{If You're Safe, Deal With It Later: Elections and fiscal policy in crises}
\author{Christopher Gandrud, Mark Hallerberg \\ \emph{Hertie School of Governance}\footnote{Please contact Christopher Gandrud
(\href{mailto:gandrud@hertie-school.org}{\nolinkurl{gandrud@hertie-school.org}}).
Our research is generously supported by the Deutsche Forschungsgemeinschaft.
All data and replication material can be found at:
\url{https://github.com/christophergandrud/EIUCrisesMeasure}.}}

\begin{document}

\maketitle


\textbf{Incomplete Working Draft}

\begin{abstract}
How do elections and electoral competitiveness affect governments' fiscal decisions during financial crises? Some previous research has found that elections have a negative affect on the fiscal costs of responding to financial crises. Politicians keep costs low to please taxpaying voters. We reexamine the relationship between elections and fiscal responses to financial crises in OECD countries using a novel approach to measuring changes in government liabilities and spending as a result of financial stress. We find evidence for a political budget cycle of responding to crises. Governments tend to take on fewer liabilities in response to crises during election years and then increase their liabilities and spending the following year. This affect is mediated by electoral competitiveness. These findings are not specific to financial crises, but output shocks in general.

\end{abstract}

[INTRODUCTION]

\section{Previous research on elections and financial crisis fiscal policy}

[POLITICAL BUDGET CYCLES]

[FISCAL RESPONSES TO CRISES]

[SOMETHING LIKE 1.3 and 1.4 FROM THE WEP PAPER]

\section{Measurement}

Accurately measuring the occurrence and intensity of financial crises as well as fiscal response to these crises is particularly difficult. In this section we describe these difficulties as well as our innovative approach to measuring overcoming them. We then discuss the right-hand variables we use to help explain these choices.

\subsection*{Measuring Fiscal Responses to Financial Crises}

A prominent source of fiscal crisis costs comes from an ongoing, though irregular IMF/World Bank data set on financial crises. The most recent version is \cite{laeven2013}, which includes a fiscal costs variable as a percentage of GDP. An earlier version of this data set was used in \cite{Keefer2007}. \cite{GandrudHallerberg2015} demonstrate that significant revisions are made to these fiscal cost estimates over time.


Responding to financial crisis often does not involve direct spending, e.g. the government giving taxpayer money directly to troubled banks to strengthen their balance sheets, but issuing new liabilities by for example lending money to banks that it borrowed. The ultimate costs of these liabilities are affected by a complex and interactive set of factors, only some of which a particular government can control. Ultimate costs can be affected by the initial size and type of the liabilities, the competency of government bureaucracies that administer them, internal and external economic developments including global liquidity shocks, and successor government decisions to change policies, such as closing a public bad bank earlier than planned possibly resulting in the assets being sold at lower prices. It is very difficult to accurately attribute costs to a particular government that develop over many years and are affected by many factors outside of the government's control. Furthermore, accounting regimes can differ significantly across time and place, such that costs for the same crisis response policy may be attributable to the government or other entities such as a bad bank \citep{gandrudHallerbergWEP}.

We take a new approach to measuring fiscal responses to financial crises. Rather than focusing on final costs, which are difficult to ascribe to choices of particular governments and may be the result of disparate accounting regimes, we focus on deviations from trend changes in government liabilities and spending. Financial and economic shocks 

\subsection*{Right-hand variables}

\bibliographystyle{apsr}
\bibliography{main.bib}

\end{document}
