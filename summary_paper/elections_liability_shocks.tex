\documentclass[]{article}
\usepackage{fullpage}
\usepackage[authoryear]{natbib}
\usepackage{setspace}
    \doublespacing
\usepackage{hyperref}
\hypersetup{
    colorlinks,
    citecolor=black,
    filecolor=black,
    linkcolor=cyan,
    urlcolor=cyan
}
\usepackage{amssymb,amsmath}
\usepackage{bm}
\usepackage{dcolumn}
\usepackage{booktabs}
\usepackage{url}
\usepackage{tikz}
\usepackage{todonotes}
\usepackage[utf8]{inputenc}
\usepackage{graphicx}
\usepackage{longtable}
\usepackage{todonotes}
\usepackage{lscape}
\usepackage{float}


\title{If You're Safe, Deal With It Later: Elections and fiscal policy in crises}
\author{Christopher Gandrud, Mark Hallerberg \\ \emph{Hertie School of Governance}\footnote{Please contact Christopher Gandrud
(\href{mailto:gandrud@hertie-school.org}{\nolinkurl{gandrud@hertie-school.org}}).
Our research is generously supported by the Deutsche Forschungsgemeinschaft.
All data and replication material can be found at:
\url{https://github.com/christophergandrud/EIUCrisesMeasure}.}}

\begin{document}

\maketitle


\textbf{Incomplete Working Draft}

\begin{abstract}
How do elections and electoral competitiveness affect governments' fiscal decisions during financial crises? Some previous research has found that elections have a negative affect on the fiscal costs of responding to financial crises. Politicians keep costs low to please taxpaying voters. We reexamine the relationship between elections and fiscal responses to financial crises in OECD countries using a novel approach to measuring changes in government liabilities and spending as a result of financial stress. We find evidence for a political budget cycle of responding to crises. Governments tend to take on fewer liabilities in response to crises during election years and then increase their liabilities and spending the following year. This affect is mediated by electoral competitiveness. These findings are not specific to financial crises, but output shocks in general.

\end{abstract}

[INTRODUCTION]

\section{Previous research on elections and financial crisis fiscal policy}

[POLITICAL BUDGET CYCLES]

[FISCAL RESPONSES TO CRISES]

[SOMETHING LIKE 1.3 and 1.4 FROM THE WEP PAPER]

\section{Measurement}

Accurately measuring the occurrence and intensity of financial crises as well as fiscal response to these crises is particularly difficult. In this section we describe these difficulties as well as our innovative approach to measuring overcoming them. We then discuss the right-hand variables we use to help explain these choices.

\subsection*{Measuring Fiscal Responses to Financial Crises}

A prominent source of fiscal crisis costs comes from an ongoing, though irregular IMF/World Bank data set on financial crises. The most recent version is \cite{laeven2013}, which includes a fiscal costs variable as a percentage of GDP. An earlier version of this data set was used in \cite{Keefer2007}. \cite{GandrudHallerberg2015} demonstrate that significant revisions are made to these fiscal cost estimates over time.

Responding to financial crisis often does not involve direct spending, e.g. the government giving taxpayer money directly to troubled banks to strengthen their balance sheets, but issuing new liabilities by for example lending money to banks that it borrowed. The ultimate costs of these liabilities are affected by a complex and interactive set of factors, only some of which a particular government can control. Ultimate costs can be affected by the initial size and type of the liabilities, the severity of the crisis, the competency of government bureaucracies that administer them, internal and external economic developments including global liquidity shocks, and successor government decisions to change policies, such as closing a public bad bank earlier than planned possibly resulting in the assets being sold at lower prices. It is very difficult to accurately attribute costs to a particular government that develop over many years and are affected by many factors outside of the government's control. Furthermore, accounting regimes can differ significantly across time and place, such that costs for the same crisis response policy may be attributable to the government or other entities such as a bad bank \citep{gandrudHallerbergWEP}.

We take a new approach to measuring fiscal responses to financial crises. Rather than focusing on final costs, which are difficult to ascribe to choices of particular governments and may be the result of disparate accounting regimes, we focus on deviations from trend changes in government liabilities and spending. This approach is based on the underlying assumption that all governments, particularly in advanced democracies--respond to economic shocks by increasing their fiscal allocations. This can be from a combination of automatic shock responses, such as unemployment insurance and deposit insurance as well as new allocations to, for example, purchase toxic assets from trouble banks or provide them with liquidity assistance. In both cases, we expect that there will be a larger fiscal response the more sever the crisis. As such we are interested in examining how political factors affect government decisions to do more (or less) than the `trend' response at a given level of crisis severity.

Before discussing the specific variables it is important to note that both our interest in policy responses in advanced democracies and data availability combine to constrict our sample to 30 OECD countries from 2003 through 2011. Please see the Online Appendix for the full list.

We estimate trend fiscal responses to financial market stress by first gathering data on general government liabilities--debt and other liabilities--and spending per country-year from the OECD.\footnote{Data was accessed through \url{https://data.oecd.org/} in June 2015.} Separate data on economic affairs spending is available, so we use that as the most relevant spending quantity. The original variables were expressed as percentages of GDP. To focus exclusively on changes to fiscal policy, rather than GDP, we transformed the variables to be in terms of the countries' 2005 GDP.\footnote{GDP data was from the OECD. Accessed June 2015.} Finally, we are primarily concerned with changes to fiscal policy, not the absolute level, which is strongly dependent on pre-shock policy choices. As such we created year-on-year change versions of our liability and spending variables.

\begin{table}
    \caption{Linear Regressions to Create Government Change in Government Liability Residuals}
    \label{liab_resid}

    \begin{center}
        
% Table created by stargazer v.5.1 by Marek Hlavac, Harvard University. E-mail: hlavac at fas.harvard.edu
% Date and time: Tue, Jun 23, 2015 - 16:16:25
\begingroup 
\tiny 
\begin{tabular}{@{\extracolsep{5pt}}lcccc} 
\\[-1.8ex]\hline 
\hline \\[-1.8ex] 
 & \multicolumn{4}{c}{\textit{Dependent variable:}} \\ 
\cline{2-5} 
\\[-1.8ex] & Liabilities & Econ. Spend & Liabilities Resid. & Econ. Spend Resid. \\ 
\\[-1.8ex] & (1) & (2) & (3) & (4)\\ 
\hline \\[-1.8ex] 
 Liabilities$_{t-1}$ & 1.063$^{***}$ &  &  &  \\ 
  & (0.027) &  &  &  \\ 
  & & & & \\ 
 Spending$_{t-1}$ &  & 0.184$^{***}$ &  &  \\ 
  &  & (0.060) &  &  \\ 
  & & & & \\ 
 Output Gap & $-$0.334$^{***}$ & $-$0.038 &  &  \\ 
  & (0.095) & (0.031) &  &  \\ 
  & & & & \\ 
 Liab. Resid.$_{t-1}$ &  &  & 0.129$^{*}$ &  \\ 
  &  &  & (0.068) &  \\ 
  & & & & \\ 
 Econ. Spend Resid.$_{t-1}$ &  &  &  & $-$0.120$^{*}$ \\ 
  &  &  &  & (0.070) \\ 
  & & & & \\ 
 Perceived Financial Stress &  &  & 7.293$^{**}$ & 2.278$^{**}$ \\ 
  &  &  & (3.011) & (1.057) \\ 
  & & & & \\ 
 Constant & $-$2.045 & 4.605$^{***}$ & $-$3.905$^{*}$ & $-$1.061 \\ 
  & (2.429) & (0.599) & (2.331) & (0.812) \\ 
  & & & & \\ 
\hline \\[-1.8ex] 
country fixed effects & Yes & Yes & Yes & Yes \\ 
\hline \\[-1.8ex] 
Observations & 329 & 296 & 270 & 242 \\ 
R$^{2}$ & 0.983 & 0.405 & 0.067 & 0.035 \\ 
Adjusted R$^{2}$ & 0.982 & 0.338 & $-$0.054 & $-$0.102 \\ 
Residual Std. Error & 4.908 (df = 297) & 1.651 (df = 265) & 5.057 (df = 238) & 1.756 (df = 211) \\ 
F Statistic & 567.145$^{***}$ (df = 31; 297) & 6.013$^{***}$ (df = 30; 265) & 0.555 (df = 31; 238) & 0.257 (df = 30; 211) \\ 
\hline 
\hline \\[-1.8ex] 
\textit{Note:}  & \multicolumn{4}{r}{$^{*}$p$<$0.1; $^{**}$p$<$0.05; $^{***}$p$<$0.01} \\ 
\end{tabular} 
\endgroup 

    \end{center}

\end{table}

Financial crises and economic growth shocks are highly related \cite[see][]{Reinhart2009}. We separated out the trend responses to economic growth shocks by first regressing the output gap\footnote{Output gap data was from the OECD. Accessed June 2015.} on changes in fiscal policies. The first and third columns of Figure~\ref{liab_resid} show coefficient estimates from these regressions. We can see that a worsening output gap is associated with increases in government liabilities. Interestingly, improving output gaps are positively associated with spending. On average governments increase their spending when the economy is doing well and increase their liabilities when the government is doing poorly.

We then took residuals from these two models and used them in a regression with a measure of perceived financial market stress. This measure is from \cite{gandrudHallEPFMS}. They conduct a textual analysis of monthly Economist Intelligence Unit (EIU) country reports to develop an index of real-time perceptions of financial market stress that they call the EIU Perceptions of Financial Market Stress (EPFMS). The Index ranges from zero--low stress--to 1--high stress. Please see \cite{gandrudHallEPFMS} for a review of other measures of financial market stress and crisis and a justification for my their measure is preferable for studying policy responses to crises. We can see in the second column of Figure~\ref{liab_resid} that perceived financial market stress is very strongly positively associated with the residuals from the output gap-liabilities regression, i.e. increases in financial market stress have an important effect on increases in government liabilities that are not explained by drops in economic output. Governments are taking on liabilities to support financial markets, not just the broader economy which may be negatively impacted by


\subsection*{Right-hand variables}

\section{Regression results}

\section*{Conclusion}




%%%%%%%%%%%% Bibliography %%%%%%%%%%%%%%
\bibliographystyle{apsr}
\bibliography{main.bib}

\section{Online Appendix}

% latex table generated in R 3.2.0 by xtable 1.7-4 package
% Thu Jun 18 16:03:50 2015
\begin{table}[ht]
\centering
{\footnotesize
\begin{tabular}{l}
  \hline
Country \\ 
  \hline
Australia \\ 
  Austria \\ 
  Belgium \\ 
  Canada \\ 
  Czech Republic \\ 
  Denmark \\ 
  Estonia \\ 
  Finland \\ 
  France \\ 
  Germany \\ 
  Greece \\ 
  Hungary \\ 
  Iceland \\ 
  Ireland \\ 
  Israel \\ 
  Italy \\ 
  Japan \\ 
  Korea, Republic of \\ 
  Luxembourg \\ 
  Netherlands \\ 
  New Zealand \\ 
  Norway \\ 
  Poland \\ 
  Portugal \\ 
  Slovakia \\ 
  Slovenia \\ 
  Spain \\ 
  Sweden \\ 
  Switzerland \\ 
  United Kingdom \\ 
   \hline
\end{tabular}
}
\end{table}


\end{document}
