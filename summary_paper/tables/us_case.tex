\begin{table}
    \caption{Portions of Selected EIU Reports for the United States (2006-2008)}
    \label{text_selections_us}
    	\begin{center}
        \begin{tabular}{c c | m{11.5cm}}
            \hline
            Month-Year & FinStress & Text Selection \\
            \hline\hline

            April 2006 & 0.13 & Corporate profitability improved remarkably in 2004-05, and this is allowing firms to fund investment from current profits. \\[0.5cm]

            April 2007 & 0.41 & A slump in the housing market is starting to have an impact on the consumer sector \ldots with liquidity in the markets for inferior quality mortgages (sub-prime and Alt-A) drying up-through stricter lending standards and rising risk aversion by more mainstream lenders\ldots \\[0.5cm]

            May 2008 & 0.5 & Financial markets have recovered somewhat in recent weeks, as market participants have increasingly swung to the belief that the worst of the financial crisis is over. Fears about the stability of the US financial system have eased particularly since the Fed supported a dramatic bail-out in mid-March of Bear Stearns, one of Wall Street's oldest and most prominent securities firms. \\[0.5cm]

            September 2008 & 0.62 & \ldots the slowing economy and rising Unemployment have made lending more risky. With house prices continuing to fall, mortgage defaults and foreclosures are rising. Delinquency rates on automotive, credit-card and student loans have also climbed. The result has been a sharp contraction in credit, imperiling consumer spending as well as the stability of weak financial institutions. \\[0.5cm]

            November 2008 & 0.71 & \ldots a large share of banks continued to tighten their lending conditions, suggesting that the economy now operates in conditions of a credit crunch \ldots \\[0.5cm]

            \hline
    \end{tabular}
    \end{center}
\end{table}
