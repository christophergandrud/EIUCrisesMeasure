\begin{tabular}{ m{2.5cm} m{1.75cm} m{6.25cm} m{2.5cm} }
    \hline
    Work & Crisis Type & Key Arguments/Findings & Crisis Data Sources \\
    \hline\hline

    %%% Bernhard and Leblang
    \cite{Bernhard2008} & Currency crisis & - Changes in the probability that cabinets will collapse condition the probability of speculative attacks.

    - Higher probability of a speculative attack decreases the probability of calling strategic elections. & Own data aggregated from multiple sources \\[0.25cm]\hline

    %%% Chwieroth and Walter
    \cite{Chwieroth2013} & Banking crises &  - Probability of government survival during crises changed over time as expectations changed about what governments should do to respond.

    - Governments with more veto players after the inter-war period are treated more harshly by voters. & \cite{ReinhartRog2010} \\[0.25cm]\hline

    \cite{CrespoTenorio2014} & Banking crisis & - Increasing globalization weakens the accountability link between politicians and voters.

    - Incumbents in open capital economies are more likely to survive a crisis, than those in closed economies. & Own data aggregated from multiple sources. \\[0.25cm]\hline

    %%% Montinola
    \cite{Montinola2003} & Banking crisis & - IMF credits decrease the probability of resolving banking crises.

    - The decisiveness of a political regime significantly influences the probability of emerging from systemic distress, though this depends on whether the crisis is moderate or severe. & Own data aggregated from multiple sources \\[0.25cm]\hline

    %%% Pepinsky
    \cite{Pepinsky2012} & Banking crisis & - Two factors--incumbent governments' responsibility for the current crisis and their responsiveness to its domestic economic effects--shape the political effects of the global economic crisis. & \cite{Laeven2010} \\[0.25cm]\hline

    \hline
\end{tabular}
