\begin{table}
    \caption{Portions of Texts EIU Reports with Country-\textbf{Maximum} FinStress values (selected countries)}
    \label{text_selections_max}
    	\begin{center}
        \begin{tabular}{c c c | m{10cm}}
            \hline
            Country & Month-Year & FinStress & Text Selection \\
            \hline\hline
            Brazil & August 2003 & 0.58 & While domestic lending to industry picked up in the second half of the year, consumer credit has been flat \ldots the effects of exchange rate depreciation on input costs in 2002 is still being felt in continued pressure on consumer prices in early 2003, requiring stringent monetary policy to suppress inflation. \\[0.5cm]

            Latvia & June 2009 & 0.65 & The economy is slowing sharply, and real GDP contracted by 18\% in the first quarter of 2009 \ldots In the current economic climate most Latvian households and companies will concentrate on reducing their debts; foreign banks operating in Latvia are also trying to reduce their exposure to Latvia by lending less. \\[0.5cm]

            Ireland & April 2011 & 0.78 & Irish households are highly indebted. Private consumption will therefore be constrained as households rebalance their balance sheets and as credit conditions remain tight in 2011-13. Investment will continue to shrink in 2012 as the collapse of the construction industry maintains momentum \ldots [the] most recent stress tests reveal the complete failure of earlier attempts to assess the impairment of the banks' balance sheets. Of particular note is the fact that no serious provision had previously been made for losses on the banks' mortgage lending, despite a massive collapse in the residential property market that has been ongoing for some years. \\[0.5cm]


            \hline
    \end{tabular}
    \end{center}
\end{table}


\begin{table}
    \caption{Portions of Texts EIU Reports with Country-\textbf{Minimum} FinStress Values (selected countries)}
    \label{text_selections_min}
    	\begin{center}
        \begin{tabular}{c c c | m{10cm}}
            \hline
            Country & Month-Year & FinStress & Text Selection \\
            \hline\hline
            Brazil & August 2005 & 0.05 & By continuing to meet the target for the primary fiscal surplus despite strong political pressure to increase spending on social programmes, and by increasing the benchmark Selic overnight rate until inflation began to subside in June, the government has confirmed its cautious stance. Apart from keeping a rein on inflation, this has helped to maintain confidence in the financial markets. \\[0.5cm]

            Latvia & March 2006 & 0.18 & The BoL [Bank of Latvia] has tried to curb the current lending boom by raising the mandatory reserve requirements for banks from 4\% to 8\% in 2005, in two steps. Further increases seem unlikely, as Latvian customers could be served from other EU countries if the costs of banking in Latvia become excessive. \\[0.5cm]

            Ireland & January 2007 & 0.06 & Private consumption will be fuelled in 2007 by strong employment growth, solid real wage increases, tax cuts, lower inflation and the maturing of a generous government-financed savings scheme \ldots Strong business confidence and a still-booming construction sector will bolster growth, but an anticipated cooling of the housing market will account for the deceleration over the outlook period. \\[0.5cm]


            \hline
    \end{tabular}
    \end{center}
\end{table}
